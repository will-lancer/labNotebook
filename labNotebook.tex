\documentclass[11pt]{article}

% basic packages
\usepackage[margin=1in]{geometry}
\usepackage[pdftex]{graphicx}
\usepackage{amsmath,amssymb,amsthm}
\usepackage{william}

% page formatting
\usepackage{fancyhdr}
\pagestyle{fancy}

\renewcommand{\sectionmark}[1]{\markright{\textsf{\arabic{section}. #1}}}
\renewcommand{\subsectionmark}[1]{}
\lhead{\textbf{\thepage} \ \ \nouppercase{\rightmark}}
\chead{}
\rhead{}
\lfoot{}
\cfoot{}
\rfoot{}
\setlength{\headheight}{14pt}

\linespread{1.03} % give a little extra room
\setlength{\parindent}{0.2in} % reduce paragraph indent a bit
\setcounter{secnumdepth}{2} % no numbered subsubsections
\setcounter{tocdepth}{2} % no subsubsections in ToC

\begin{document}

% make title page
\thispagestyle{empty}
\bigskip \
\vspace{0.1cm}

\begin{center}
{\fontsize{22}{22} \selectfont Lab Notebook for}
\vskip 16pt
{\fontsize{36}{36} \selectfont \bf \sffamily PHY 445}
\vskip 24pt
	{\fontsize{18}{18} \selectfont \rmfamily \href{https://will-lancer.github.io}{Will Lancer}} 
\vskip 6pt
{\fontsize{14}{14} \selectfont \ttfamily will.m.lancer@gmail.com} 
\vskip 24pt
\end{center}

{\parindent0pt \baselineskip=15.5pt}
\noin
Lab notebook for PHY 445 Spring 2026

% make table of contents
\newpage
\microtoc
\newpage

% main content
\section{Feb 5, 2026}

\textbf{Link to the lab website:} \url{https://you.stonybrook.edu/phy445/experiment-overview/}\\

\noin
\textbf{Note:} We are allowed to have 1--2 extra days for this experiment as there was no lab manual given.

\subsection{Experiment: Nuclear Magnetic Resonance}

Experiments that we will run:

\begin{itemize}
    \item We will test $T_1$, $T_2$, $T_2^*$, and $T_2'$ for three samples: distilled water, 
    tap water, and water with metal in it (or honey if we do not have that);
\end{itemize}

\subsubsection{Sub-experiment one: distilled water}

We got the distilled water from the HEP-ex water room; poured it into the little pipe 
without a pipette because the only one we found was super dirty and probably full of
AIDS.


\paragraph{Distilled water: $T_1$}

Experimental set-up and tuning for measuring $T_1$ for distilled water:

\begin{itemize}
    \item RF frequency matching: Tuned it until there was a minimal sine curve. Measured frequency is $21.01291~\mathrm{MHz}$.
    \item $\pi/2$ pulse: $1.12 \pm 0.005 \, \mu \mathrm{s}$
    \item $\pi$ pulse: $7.32 \pm 0.005 \, \mu \mathrm{s}$
\end{itemize}

Now we tried to measure $T_1$ for distilled water, but this was very hard:
\begin{itemize}
    \item We didn't know how to export the osc. data to fit it.
    \item We didn't know if the pulse was just one peak of the sine
    curve or if it was the entire curve.
    \item Corliss came over and helped us kind of. We didn't know how
    to do anything. He pointed out that we were syncing on $A$ when we
    probably should have been syncing on $B$.
    \item TA came over and helped us more; got back to my original
    question which I'm just going to email Prof. Liu.
    \item After class, emailed Prof. Liu the following:
    \begin{quote}
        Hi Professor Liu,\\

        My partner and I are trying to measure the $T_1$ timescale for the NMR
        experiment in PHY 445. We know that we have to fit a damped sine curve to the
        response of the nucleus to the A pulse to determine $T_1$. After asking both
        the TA and Professor Corliss for advice on how to do this, we are still not
        sure how to take the oscilloscope data and translate it into a form that we
        can fit a curve to. Do you have any advice on how to do this?\\

        Also, you previously said that some sort of iron solution is commonly used
        in this experiment to test different responsivities to the external magnetic
        field and RF pulses. Will this be made available to us next Tuesday?\\

        Thanks,\\
        Will Lancer
    \end{quote}
\end{itemize}

\subsubsection{Sub-experiment Two: Tap Water}

\subsubsection{Sub-experiment Three: Water with Metal in It}


\end{document}