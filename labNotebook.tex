\documentclass[11pt]{article}

% basic packages
\usepackage[margin=1in]{geometry}
\usepackage[pdftex]{graphicx}
\usepackage{amsmath,amssymb,amsthm}
\usepackage{william}

% page formatting
\usepackage{fancyhdr}
\pagestyle{fancy}

\renewcommand{\sectionmark}[1]{\markright{\textsf{\arabic{section}. #1}}}
\renewcommand{\subsectionmark}[1]{}
\lhead{\textbf{\thepage} \ \ \nouppercase{\rightmark}}
\chead{}
\rhead{}
\lfoot{}
\cfoot{}
\rfoot{}
\setlength{\headheight}{14pt}

\linespread{1.03} % give a little extra room
\setlength{\parindent}{0.2in} % reduce paragraph indent a bit
\setcounter{secnumdepth}{2} % no numbered subsubsections
\setcounter{tocdepth}{2} % no subsubsections in ToC

\begin{document}

% make title page
\thispagestyle{empty}
\bigskip \
\vspace{0.1cm}

\begin{center}
{\fontsize{22}{22} \selectfont Lab Notebook for}
\vskip 16pt
{\fontsize{36}{36} \selectfont \bf \sffamily PHY 445}
\vskip 24pt
	{\fontsize{18}{18} \selectfont \rmfamily \href{https://will-lancer.github.io}{Will Lancer}} 
\vskip 6pt
{\fontsize{14}{14} \selectfont \ttfamily will.m.lancer@gmail.com} 
\vskip 24pt
\end{center}

{\parindent0pt \baselineskip=15.5pt}
\noin
Lab notebook for PHY 445 Spring 2026

% make table of contents
\newpage
\microtoc
\newpage

% main content
\section{Feb 5, 2026}

\textbf{Link to the lab website:} \url{https://you.stonybrook.edu/phy445/experiment-overview/}\\

\noin
\textbf{Note:} We are allowed to have 1--2 extra days for this experiment as there was no lab manual given.

\subsection{Experiment: Nuclear Magnetic Resonance}

Experiments that we will run:

\begin{itemize}
    \item We will test $T_1$, $T_2$, $T_2^*$, and $T_2'$ for three samples: distilled water, 
    tap water, and water with metal in it (or honey if we do not have that);
\end{itemize}

\subsubsection{Sub-experiment one: distilled water}

We got the distilled water from the HEP-ex water room; poured it into the little pipe 
without a pipette because the only one we found was super dirty and probably full of
AIDS.


\paragraph{Distilled water: $T_1$}

Experimental set-up and tuning for measuring $T_1$ for distilled water:

\begin{itemize}
    \item RF frequency matching: Tuned it until there was a minimal sine curve. Measured frequency is $21.01291~\mathrm{MHz}$.
    \item $\pi/2$ pulse: $1.12 \pm 0.005 \, \mu \mathrm{s}$
    \item $\pi$ pulse: $7.32 \pm 0.005 \, \mu \mathrm{s}$
\end{itemize}

Now we tried to measure $T_1$ for distilled water, but this was very hard:
\begin{itemize}
    \item We didn't know how to export the osc. data to fit it.
    \item We didn't know if the pulse was just one peak of the sine
    curve or if it was the entire curve.
    \item Corliss came over and helped us kind of. We didn't know how
    to do anything. He pointed out that we were syncing on $A$ when we
    probably should have been syncing on $B$.
    \item TA came over and helped us more; got back to my original
    question which I'm just going to email Prof. Liu.
    \item After class, emailed Prof. Liu the following:
    \begin{quote}
        Hi Professor Liu,\\

        My partner and I are trying to measure the $T_1$ timescale for the NMR
        experiment in PHY 445. We know that we have to fit a damped sine curve to the
        response of the nucleus to the A pulse to determine $T_1$. After asking both
        the TA and Professor Corliss for advice on how to do this, we are still not
        sure how to take the oscilloscope data and translate it into a form that we
        can fit a curve to. Do you have any advice on how to do this?\\

        Also, you previously said that some sort of iron solution is commonly used
        in this experiment to test different responsivities to the external magnetic
        field and RF pulses. Will this be made available to us next Tuesday?\\

        Thanks,\\
        Will Lancer
    \end{quote}
\end{itemize}

\section{Feb 10, 2026}

\begin{itemize}
    \item Prof. Liu responded:
    \begin{quote}
        Hi Will,

        Prof. Xu Du is taking care of the NMR measurement this semester. You
        probably confused me as Prof. Du?

        Regarding the data, usually you can take the oscilloscope data from a USB
        drive and import it into the computer. Then you can fit it from there.
        (Depending on which oscilloscope you are using this semester.)

        Everything should be in the test kit -- usually I will suggest you use the
        mineral oil first, and then the rubber.
    \end{quote}
    \item Whoops! Literally everyone in the lab thought that Prof. Du was Prof. Liu,
    so that's a bit funny.
    \item Anyway, time to continue with this experiment.
\end{itemize}

Ok, we're going to switch our experiemnt to mineral oil on Prof. Liu's
recommendation.

\subsection{Sub-experiment: mineral oil}

We're using the light mineral oil. Time to re-measure all of our stuff!
\begin{itemize}
    \item RF frequency matching: $21.11319 \pm 0.00001 \, \mathrm{MHz}$.
    \item $\pi/2$ pulse: $4.48 \pm 0.005 \, \mu \mathrm{s}$.
    \item $\pi$ pulse: $9.12 \pm 0.005 \, \mu \mathrm{s}$.
\end{itemize}

We think that we want an $A$ pulse and then a $B$ pulse. The former being
a $\pi$ pulse and the latter being a $\pi/2$ pulse. But we don't really see
that on the oscilloscope. We varied tons of numbers, and then figured out that
we can put just one $B$ pulse. We did that, but when we vary $\tau$, we don't
see the amplitude of $B$ change; this isn't what we expected, as we'd expect the
curve to follow the equation $\sim 1 - e^{-\tau/T_1}$.\\

\subsection{Enter Professor Du}

We called Prof. Du over and he determined after some debugging that
we were way off our resonance frequency. We were using the previous
value, but he determined that it should be $21.00854 \pm 0.00001 \, \mathrm{MHz}$.
The $A$ pulse and $B$ pulses were wrong because we were off resonance:
the new ones are $A = 6.72 \pm 0.005 \, \mu \mathrm{s}$ and $B = 2.60 \pm 0.005 \, \mu \mathrm{s}$.
\begin{align*}
    \textbf{NOTE: ALL OF THESE VALUES ARE FOR BEESWAX, NOT MINERAL OIL.}
\end{align*}
The lesson from this is that the vertical scale of the blue line was \emph{super}
zoomed out; because of this, I tuned it to what seeemd tiniest but it was wayyyy
off. So, make your vertical and horizontal scales fit the scales at hand or else
you will get cooked.



\end{document}