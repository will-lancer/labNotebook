\documentclass[11pt]{article}

% basic packages
\usepackage[margin=1in]{geometry}
\usepackage[pdftex]{graphicx}
\usepackage{amsmath,amssymb,amsthm}
\usepackage{william}

% page formatting
\usepackage{fancyhdr}
\pagestyle{fancy}

\renewcommand{\sectionmark}[1]{\markright{\textsf{\arabic{section}. #1}}}
\renewcommand{\subsectionmark}[1]{}
\lhead{\textbf{\thepage} \ \ \nouppercase{\rightmark}}
\chead{}
\rhead{}
\lfoot{}
\cfoot{}
\rfoot{}
\setlength{\headheight}{14pt}

\linespread{1.03} % give a little extra room
\setlength{\parindent}{0.2in} % reduce paragraph indent a bit
\setcounter{secnumdepth}{2} % no numbered subsubsections
\setcounter{tocdepth}{2} % no subsubsections in ToC

\begin{document}

% make title page
\thispagestyle{empty}
\bigskip \
\vspace{0.1cm}

\begin{center}
{\fontsize{22}{22} \selectfont Lab Notebook for}
\vskip 16pt
{\fontsize{36}{36} \selectfont \bf \sffamily PHY 445}
\vskip 24pt
	{\fontsize{18}{18} \selectfont \rmfamily \href{https://will-lancer.github.io}{Will Lancer}} 
\vskip 6pt
{\fontsize{14}{14} \selectfont \ttfamily will.m.lancer@gmail.com} 
\vskip 24pt
\end{center}

{\parindent0pt \baselineskip=15.5pt}
\noin
Lab notebook for PHY 445 Spring 2026

% make table of contents
\newpage
\microtoc
\newpage

% main content
\section{Feb 5, 2026}

\textbf{Link to the lab website:} \url{https://you.stonybrook.edu/phy445/experiment-overview/}\\

\noin
\textbf{Note:} We are allowed to have 1--2 extra days for this experiment as there was no lab manual given.

\subsection{Experiment: Nuclear Magnetic Resonance}

Experiments that we will run:

\begin{itemize}
    \item We will test $T_1$, $T_2$, $T_2^*$, and $T_2'$ for three samples: distilled water, 
    tap water, and water with metal in it (or honey if we do not have that);
\end{itemize}

\subsubsection{Sub-experiment one: distilled water}

We got the distilled water from the HEP-ex water room; poured it into the little pipe 
without a pipette because the only one we found was super dirty and probably full of
AIDS.


\paragraph{Distilled water: $T_1$}

Experimental set-up and tuning for measuring $T_1$ for distilled water:

\begin{itemize}
    \item RF frequency matching: Tuned it until there was a minimal sine curve. Measured frequency is $21.01291~\mathrm{MHz}$.
    \item $\pi/2$ pulse: $1.12 \pm 0.005 \, \mu \mathrm{s}$
    \item $\pi$ pulse: $7.32 \pm 0.005 \, \mu \mathrm{s}$
\end{itemize}

Now we tried to measure $T_1$ for distilled water, but this was very hard:
\begin{itemize}
    \item We didn't know how to export the osc. data to fit it.
    \item We didn't know if the pulse was just one peak of the sine
    curve or if it was the entire curve.
    \item Corliss came over and helped us kind of. We didn't know how
    to do anything. He pointed out that we were syncing on $A$ when we
    probably should have been syncing on $B$.
    \item TA came over and helped us more; got back to my original
    question which I'm just going to email Prof. Liu.
    \item After class, emailed Prof. Liu the following:
    \begin{quote}
        Hi Professor Liu,\\

        My partner and I are trying to measure the $T_1$ timescale for the NMR
        experiment in PHY 445. We know that we have to fit a damped sine curve to the
        response of the nucleus to the A pulse to determine $T_1$. After asking both
        the TA and Professor Corliss for advice on how to do this, we are still not
        sure how to take the oscilloscope data and translate it into a form that we
        can fit a curve to. Do you have any advice on how to do this?\\

        Also, you previously said that some sort of iron solution is commonly used
        in this experiment to test different responsivities to the external magnetic
        field and RF pulses. Will this be made available to us next Tuesday?\\

        Thanks,\\
        Will Lancer
    \end{quote}
\end{itemize}

\section{Feb 10, 2026}

\begin{itemize}
    \item Prof. Liu responded:
    \begin{quote}
        Hi Will,

        Prof. Xu Du is taking care of the NMR measurement this semester. You
        probably confused me as Prof. Du?

        Regarding the data, usually you can take the oscilloscope data from a USB
        drive and import it into the computer. Then you can fit it from there.
        (Depending on which oscilloscope you are using this semester.)

        Everything should be in the test kit -- usually I will suggest you use the
        mineral oil first, and then the rubber.
    \end{quote}
    \item Whoops! Literally everyone in the lab thought that Prof. Du was Prof. Liu.
    \#rip. \#duNotLiu.
    \item Anyway, time to continue with this experiment.
\end{itemize}

Ok, we're going to switch our experiment to mineral oil on Prof. Liu's
recommendation.

\subsection{Sub-experiment: mineral oil}

We're using the light mineral oil. Time to re-measure all of our stuff!
\begin{itemize}
    \item RF frequency matching: $21.11319 \pm 0.00001 \, \mathrm{MHz}$.
    \item $\pi/2$ pulse: $\note{didn't take for some reason?} \pm 0.005 \, \mu \mathrm{s}$.
    \item $\pi$ pulse: $6.76 \pm 0.005 \, \mu \mathrm{s}$.
\end{itemize}

We think that we want an $A$ pulse and then a $B$ pulse. The former being
a $\pi$ pulse and the latter being a $\pi/2$ pulse. But we don't really see
that on the oscilloscope. We varied tons of numbers, and then figured out that
we can put just one $B$ pulse. We did that, but when we vary $\tau$, we don't
see the amplitude of $B$ change; this isn't what we expected, as we'd expect the
curve to follow the equation $\sim 1 - e^{-\tau/T_1}$.\\

\subsection{Enter Professor Du, the King of NMR™ \du}

We called Prof. Du over and he determined after some debugging that
we were way off our resonance frequency. We were using the previous
value, but he determined that it should be $21.00854 \pm 0.00001 \, \mathrm{MHz}$.
The $A$ pulse and $B$ pulses were wrong because we were off resonance:
the new ones are $A = 6.72 \pm 0.005 \, \mu \mathrm{s}$ and $B = 2.60 \pm 0.005 \, \mu \mathrm{s}$.\\

\begin{centering}
    \textbf{NOTE: ALL OF THESE VALUES ARE FOR BEESWAX, NOT MINERAL OIL.\\
    Xu ``The Goat of NMR'' Du switched us to that material in his infinite wisdom.}
\end{centering}\\

The lesson from this is that the vertical scale of the blue line was \emph{super}
zoomed out; because of this, I tuned it to what seeemd tiniest but it was wayyyy
off. So, make your vertical and horizontal scales fit the scales at hand or else
you will get cooked.

Great. I then was curious about what the period was and why moving it around
shifted the amplitude of the $B$ pulse, so we called over Prof. Du again
and he explained it to us. We then increased the period from $100 \mathrm{ms}$
to $600 \mathrm{ms}$ because the relaxation time of oil is kinda long.

After switching the period, I noticed the frequency was looking sus
and a little bit beatastic, so we changed it again to make it not beating:

\begin{itemize}
    \item RF frequency matching: $21.00740 \pm 0.00001 \, \mathrm{MHz}$.
    \item $\pi$ pulse: $6.76 \pm 0.005 \, \mu \mathrm{s}$.
    \item $\pi/2$ pulse: $3.36 \pm 0.005 \, \mu \mathrm{s}$.
\end{itemize}

\section{Feb 12, 2026}

Got into lab. Eric and friends did some bullshit to the scope and
the whatever frequenct setter thing. We fixed it, and switched the
RF frequency to: $21.01509 \pm 0.00001 \, \mathrm{MHz}$.

We are now going to determine $T_1$ for the light mineral oil.
See Alo's lab notebook for the relevant table.

\subsection{Beeswax $T_1$ measurement}

Remeasured the relevant values:
\begin{itemize}
    \item RF frequency matching: $21.01183 \pm 0.00001 \, \mathrm{MHz}$.
    \item $\pi$ pulse: 
    \item $\pi/2$ pulse:
\end{itemize}

\subsection{Nevermind, fml}

We realized that our previous $A$ and $B$ numbers were for BEESWAX,
NOT MINERAL OIL. So we need to retake our data!!! So fun! Experiments
are just so wonderful and not absolutely shit, I just love them! This
definitely isn't wasting my life!\\

\noin
Relevant values:
\begin{itemize}
    \item RF: $21.01155 \pm 0.00001 \, \mathrm{MHz}$.
    \item $\pi$: $7.36 \pm 0.005 \, \mu \mathrm{s}$.
    \item $\pi/2$: $3.68 \pm 0.005 \, \mu \mathrm{s}$.
\end{itemize}

Asked my best buddy ChatGPT to plot the table for me. He gave us these
tables:
\begin{figure}[H]
    \centering
    \includegraphics[width=0.8\textwidth]{assets/output-5.png}
    \caption{chungus 1}
    \label{fig:beeswax_table}
\end{figure}

\begin{figure}[H]
    \centering
    \includegraphics[width=0.8\textwidth]{assets/output-6.png}
    \caption{chungus 2}
    \label{fig:beeswax_table_2}
\end{figure}


\subsection{Beeswax, pt. 2}

RV:
\begin{itemize}
    \item RF: $21.00852 \pm 0.00001 \, \mathrm{MHz}$.
    \item $\pi$: $6.80 \pm 0.005 \, \mu \mathrm{s}$.
    \item $\pi/2$: $3.46 \pm 0.005 \, \mu \mathrm{s}$.
    \item Period: $700 \pm 0.5 \, \mathrm{ms}$.
\end{itemize}

See Alo's table for the raw data.

\begin{figure}[H]
    \centering
    \includegraphics[width=0.8\textwidth]{assets/output-7.png}
    \caption{chungus 3}
    \label{fig:beeswax_table}
\end{figure}

\begin{figure}[H]
    \centering
    \includegraphics[width=0.8\textwidth]{assets/output-8.png}
    \caption{chungus 4}
    \label{fig:beeswax_table_2}
\end{figure}

\section{Feb 17, 2026}

Let's chung our way through this day$\ldots$ this class genuinely
sucks. Whatever. Let's do the spin-echo experiment.

\subsection{Spin-echo: light mineral oil}

\begin{itemize}
    \item RF: $21.015224 \pm 0.00001 \, \mathrm{MHz}$.
    \item $\pi$: $7.62 \pm 0.005 \, \mu \mathrm{s}$.
    \item $\pi/2$: $3.98 \pm 0.005 \, \mu \mathrm{s}$.
    \item Period: $600 \pm 0.5 \, \mathrm{ms}$.
\end{itemize}

Señor Du helped us with these values.

Now we are going to do the spin-echo experiment. This
procedure goes as follows:
\begin{enumerate}
    \item Apply a $\pi/2$ pulse: this is the $A$ pulse.
    \item Apply a million $\pi$ pulses: these are the $B$ pulses.
    \item The spin-echo curve envelope will go like $e^{-t/T_2'}$;
    you just extract this data and fit it to a curve to get $T_2'$.
\end{enumerate}

We will use $30$ $\pi$-pulses and fit the curve to get $T_2'$.
I think we will have to use thumbdrive to export the data on the
scope for this one---just taking the data points may not be enough,
or at least that's not what the other groups did.

We got this epic data from the spin-echo experiment:
\begin{figure}[H]
    \centering
    \includegraphics[width=0.8\textwidth]{assets/NewFileALO.png}
    \caption{Spin-echo raw data for light mineral oil}
    \label{fig:spin-echo-raw-data}
\end{figure}


\begin{figure}[H]
    \centering
    \includegraphics[width=0.8\textwidth]{assets/spinEchoLightMineralOil.png}
    \caption{Spin-echo data for light mineral oil}
    \label{fig:spin_echo_data}
\end{figure}

We are the goats of NMR.

\subsection{Beeswax-maxxing}

Relevant values:
\begin{itemize}
    \item RF: $21.01583 \pm 0.00001 \, \mathrm{MHz}$.
    \item $\pi/2$: $7.62 \pm 0.005 \, \mu \mathrm{s}$.
    \item $\pi$: $3.98 \pm 0.005 \, \mu \mathrm{s}$.
    \item Period: $700 \pm 0.5 \, \mathrm{ms}$.
\end{itemize}

Tables type shit!

\begin{figure}[H]
    \centering
    \includegraphics[width=0.8\textwidth]{assets/NewFile1AACC.png}
    \caption{Spin-echo raw data for beeswax}
    \label{fig:spin-echo-raw-data}
\end{figure}

\begin{figure}[H]
    \centering  
    \includegraphics[width=0.8\textwidth]{assets/spinEchoBeeswax.png}
    \caption{Spin-echo data for beeswax}
    \label{fig:spin-echo-beeswaxmaxxing}
\end{figure}

We are the goats of NMR, part 2.

\section{Feb 19, 2026}

Back at it again. We just need to measure $T_2$ today, and then
we are free from the shackles of NMR. The midlab report is due
today; Alo submitted it for us. She now knows how to use Git,
which is great.

I believe the way you measure $T_2$ is by pulsing a $\pi/2$ pulse
and just waiting for the voltage to decay back to zero. Then you fit
that to $e^{-t/T_2}$ and you're done. Let's check with Du to see if that's
Tru.

We checked with Du: this is indeed Tru. Time to Mu-ve on!

\subsection{Light mineral oil: $T_2$}

The ppl we share this experiment with messed
up the stuff on the scope and with the light mineral
oil sample. We now have to clean up their mess.

Relevant values:
\begin{itemize}
    \item RF: $21.01596 \pm 0.00001 \, \mathrm{MHz}$.
    \item $\pi/2$: $4.08 \pm 0.005 \, \mu \mathrm{s}$.
    \item Period: $600$
\end{itemize}

We had to put the ring on the sample quite low to avoid a big shoulder.
This wasn't the case before, but the other people on this experiment
completely messed up the same and didn't store it correctly, so the
light mineral oil wasn't concentrated entirely at the bottom of the
tube. This lowered the measured voltage, but shouldn't be a big deal.

We got this picture from the scope, and this curve was fit to it
by my bestie ChatGPT 5.2 Extended Thinking. Samriddhi was goated
and let us use her personal flashdrive to get this data, and the
previous data.

\begin{figure}[H]
    \centering
    \includegraphics[width=0.8\textwidth]{assets/rawT2datalightmineraloil.png}
    \caption{Raw $T_2$ data for light mineral oil}
    \label{fig:raw_t_2_data_light_mineral_oil}
\end{figure}

\begin{figure}[H]
    \centering
    \includegraphics[width=0.8\textwidth]{assets/fittedt2lightmineraloil.png}
    \caption{$T_2$ data for light mineral oil, fit to $e^{-t/T_2}$ curve. $T_2 = 2.15 \pm 0.05 \, \mathrm{ms}$.}
    \label{fig:t_2_data_light_mineral_oil}
\end{figure}

\subsection{Beeswax: $T_2$}

Relevant values:
\begin{itemize}
    \item RF: $21.01245 \pm 0.00001 \, \mathrm{MHz}$.
    \item $\pi/2$: $4.10 \pm 0.005 \, \mu \mathrm{s}$.
    \item Period: $700 \pm 0.5 \, \mathrm{ms}$.
\end{itemize}

Note that each horizontal block on the raw data below is 50 microseconds long.

\begin{figure}[H]
    \centering
    \includegraphics[width=0.8\textwidth]{assets/rawT2dataBeeswax.png}
    \caption{Raw $T_2$ data for beeswax}
    \label{fig:raw_t_2_data_beeswax}
\end{figure}

\begin{figure}[H]
    \centering
    \includegraphics[width=0.8\textwidth]{assets/fittedt2Beeswax.png}
    \caption{$T_2$ data for beeswax, fit to $e^{-t/T_2}$ curve}
    \label{fig:t_2_data_beeswax}
\end{figure}

\end{document}